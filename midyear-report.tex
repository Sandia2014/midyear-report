%%% Template File for Use with the hmcclinic.cls.
%%%
%%% C.M. Connelly <cmc@math.hmc.edu>
%%%  $Id: clinic-template.tex 368 2011-08-01 18:53:50Z cmc $
%%%
%%%  Tag: $Name$


%%% !!! HMC STUDENTS SHOULD REMOVE THE FOLLOWING COPYRIGHT NOTICE FROM
%%% !!! FINAL SUBMISSIONS.

%%% Copyright (C) 2004-2010 Department of Mathematics, Harvey Mudd College.

%%% See the COPYING document, which should accompany this
%%% distribution, for information about distribution and modification
%%% of the document and its components.

%%% !!! END COPYRIGHT NOTICE.

%%%%%%%%%%%%%%%%%%%%%%%%%%%%%%%%%%%%%%%%%%%%%%%%%%%%%%%%%%%%%%%%%%%%%%
%%% Note that for your Clinic report, you should remove any        %%%
%%% comments that aren't relevant to your report.  You should also %%%
%%% remove the copyright notice assigning copyright to the         %%%
%%% department (depending on your sponsor, you may need to assign  %%%
%%% copyright to the sponsoring organization.                      %%%
%%%%%%%%%%%%%%%%%%%%%%%%%%%%%%%%%%%%%%%%%%%%%%%%%%%%%%%%%%%%%%%%%%%%%%

%%% Preamble.

%%% The top part of your document is called the preamble.  You supply
%%% some basic information about the document (such as its title and
%%% author) in a form that LaTeX can understand here.


%%% The first active line in your LaTeX document is the \documentclass
%%% command, which loads a LaTeX class file.  Class files generally
%%% define the appearance of a document, and include a variety of
%%% structural commands.

%%% Clinic reports use the clinic class, which should be located
%%% somewhere in TeX's search path.

%%% For midyear reports, include the midyear option, as in
%%%   \documentclass[midyear]{hmcclinic}
\documentclass{hmcclinic}

%%% You can also load additional LaTeX packages, or style files, that
%%% affect the way that the document is laid out, typeset, or supply
%%% additional commands or environments.  If you choose to load
%%% additional packages, make sure that they appear *before* the
%%% line loading hyperref; hyperref changes pieces of other
%%% packages, so it's important that it be loaded last.

% \usepackage{graphicx}           % More control over graphic inclusion.
% \usepackage{amsthm}             % AMS theorem styles


%%% Load all other packages before this point.

%%% Load hyperref.
\usepackage[breaklinks=true,
  bookmarks,
  pdfpagemode=UseOutlines,
  pdfpagelayout=SinglePage]{hyperref}


%%% The preamble can also be used to define your own commands and
%%% environments, set some constants that will be used throughout your
%%% document, and so on.

%%% As you may have guessed, LaTeX's comment character is the percent
%%% sign.  Any line that starts with a % will be ignored.  You can
%%% also use the comment character to add comments to the end of a
%%% line that will be parsed by TeX.

%%% The optional \includeonly command allows you to specify the names
%%% of chapters that you want to typeset.  It is useful for debugging
%%% or for working intensely on one particular part of your document
%%% when you don't want to take the time to retypeset the entire document.


%%% This optional command provides additional context around an error.
%%% It can be helpful when tracking down a problem. 
%\setcounter{errorcontextlines}{1000}


%%% Information about this document.

%%% I find it most useful to put identifying information about a
%%% document near the top of the preamble.  Technically, this
%%% information must precede the \maketitle command, which often
%%% appears immediately after the beginning of the document 
%%% environment.  Placing it near the top of the document makes it
%%% easier to identify the document, and keeps it from getting
%%% mixed up with the content of your document.

%%% So, some questions.

%% What is the name of the company or organization sponsoring your project?
\sponsor{Sandia National Laboratories}

%% What is the title of your report?
\title{Midyear Update}

%% Who are the authors of the report (your team members)?  (Separate
%% names with \and.)
\author{Alex Gruver \and Ellen Hui \and Tyler Marklyn \and Brett Collins~(Project Manager)}

%% What is your faculty advisor's name?  (Again, separate names with
%% \and, if necessary.)
\advisor{Jeff Amelang}

%% Liaison's name or names?
\liaison{Carter Edwards}

%% By not specifying a date with the \date command, the date the
%% document is typeset will be added.

%% If you need to put in a specific date, do so with
%%  \date{May 13, 2004}
%% You probably shouldn't, however.

%%% End of information section.


%%% New commands and environments.

%%% You can define your own commands and environments here.  If you
%%% have a lot of material here, you might want to consider splitting
%%% the commands and environments into a separate ``style'' file that
%%% you load with \usepackage.

% \newcommand{\coolcommand}[1]{#1 is cool.} % Lets everyone know that
                                % the person or thing that you provide
                                % as the argument to the command is
                                % cool.

% \newcounter{cms}


%%% Some theorem-like command definitions.

%%% The \newtheorem command comes from the amsthm package.  That
%%% package is *not* loaded by the class file, so if you choose
%%% to use these commands, you'll need to load the package above.

% \newtheorem{thm}{Theorem}[chapter]
% \newtheorem{lem}{Lemma}[chapter]


%%% If you find that some words in your document are being hyphenated
%%% incorrectly, you can specify the correct hyphenation using the
%%% \hyphenation command.  Note that words are separated by
%%% whitespace, as shown below.

\hyphenation{ap-pen-dix wer-ther-i-an}


%%% The start of the document!

%% The document environment is the main environment in any LaTeX
%% document.  It contains other environments, as well as your text.

\begin{document}

%%% The front matter of a large document includes the title page or
%%% pages, tables of contents, lists of figures or tables, and so on,
%%% your abstract, a preface or introduction, and so on.  It's
%%% delineated with the \frontmatter command.



%%% One of the things that the \frontmatter does is make page
%%% numbers appear as lowercase Roman numerals---i, vi, xii, and so
%%% on.

%%% The first thing in the front matter is your title page.  The title
%%% page is formatted by commands in the document class file, so you
%%% don't need to worry about what it looks like -- just putting the
%%% \maketitle command in your document (and filling in the necessary
%%% information for the identification commands above) is enough.

\maketitle

\mainmatter


%%% Content.

% FIXME: all section headers are placeholders, we should put something more
% official in their place.

\section*{What went well}
% Alex's section
This semester we have been able to export multiple kernels from Intrepid and 
get them working with Kokkos. While we have not been able to get speedup
on the order of one hundred times faster so far, we have learned a lot about
using Kokkos, which was one of the most important aspects of the project, as well
as GPU programming in general. Additionally, we do see speedup on these
kernels for Sandia's use case, which involves repeated application of the kernels
to data stored on the GPU. This is a boon for us in terms of speedup because it 
allows us to avoid expensive data copies from the GPU to the CPU.

We have also worked through a short online class as well as many example 
problems to sharpen our GPU programming skills. Optimizing code for GPU performance
is a career, but by tackling example problems before we moved on to intrepid
we were able to familiarize ourselves with GPU programming enough to make meaningful 
contributions to the code.

\section*{What didn't go well}
% Ellen's section
Our initial plan for this semester called for a fully re-engineered Intrepid
package using Kokkos by early December, which we would present in a seminar
during our site visit to Sandia at the end of the semester.  However, we have
run into a few setbacks, and are therefore unlikely to meet that goal.

Originally, we had hoped to find a working algorithm using Kokkos that yielded
speedup comparable to hand-coded Cuda and OpenMP for Intrepid tensor
contractions.  However, our prototype Kokkos implementations have lagged in
speed performance, even when taking into account coalescence and memory layout.
We have since abandoned Kokkos multidimensional Views for the moment, instead
using single-dimension Kokkos views in order to have finer control over the
memory layout.  While this approach is yielding promising results with Kokkos
OpenMP, both the Kokkos Cuda and manual Cuda implementations have been
disappointingly slow.

We are continuing to work on getting speedup from Kokkos and manual Cuda, but
we feel it would be counterproductive to work on more Intrepid kernels until we
have our single prototype kernel producing performance gains on the expected
order of magnitude.  Therefore, given the amount of time left in the semester,
it is likely that we will have to settle for presenting our prototype kernel
instead of a fully reworked Intrepid package in December. 

\section*{Future Plans}
% Brett's section
Finishing the Kokkos versions of Intrepid's ArrayTools functions is our prority going into the
spring.  We were originally hoping to finish this task by Winter Break (December 16th), but we have 
come across unexpected obstacles as previously mentioned. The way that we plan on finishing 
this job is by focusing our efforts on a couple of the simpler methods in the ArrayTools library 
in order to get the desired speed-up. Once we have correct Kokkos implementations for a couple of
the functions we can apply similar parallel/memory access algorithms to speed-up
the remaining functions in Intrepid. Fortunately, the task of our clinic is to show a proof of concept 
for Kokkos, so it's not critical for us to spend time overhauling the entire Intrepid kernel. 
If we have a couple functions  with great speedup using Kokkos, then that should be enough
to satisfy the purpose of our project. 

Throughout this entire process it is important for  us to continue to record our thoughts about Kokkos
and put these notes on our team wiki so that our liaisons have access to them. 
This will give Sandia valuable feedback about Kokkos and valuable information on how to improve the
user's experience. Logging these notes on the wiki is another of our priorities, since we 
serve a valuable function as Kokkos beta testers. 

Depending on the upcoming work and whether we have a breakthrough in speedup, we may meet our 
Winter Break deadline or finish by mid February. We do not yet know what our task after completing
Intrepid will be, but it will most likely be creating a Kokkos version of a different high 
performance computing kernel. Depending on how similar the new kernel is to Intrepid in terms of
functionality, we plan to tackle the problem as follows. We will begin by finding the
simplist functions in the kernel, then create Kokkos versions of those functions, manipulate memory accesses and 
parallelization points until we achieve reasonable speedup, then apply the algorithm to the rest 
of the functions. Our goal is to finish the Intrepid kernel as well as
at least one more kernel. 

Our stretch goals are to continue to create Kokkos versions of more high
performance computing kernels or integrating the Kokkos versions of the kernels back into the 
original packages. We are leaning towards continuing to create Kokkos versions of more kernels 
because we will have a better understanding of Kokkos and what is required to gain speed-up. 
Integrating the Kokkos versions of the kernels into their original packages brings up new 
challenges that we have not yet faced and will have a much greater learning curve. Overall, our 
priorities, in order, are: finish a couple functions within Intrepid ArrayTools with expected 
speed-up to show a Kokkos proof of concept, log our experiences with Kokkos on the team wiki, 
finish creating a Kokkos version of Intrepid, then repeat with a new high performance computing kernel.

\end{document}


