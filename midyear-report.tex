%%% Template File for Use with the hmcclinic.cls.
%%%
%%% C.M. Connelly <cmc@math.hmc.edu>
%%%  $Id: clinic-template.tex 368 2011-08-01 18:53:50Z cmc $
%%%
%%%  Tag: $Name$


%%% !!! HMC STUDENTS SHOULD REMOVE THE FOLLOWING COPYRIGHT NOTICE FROM
%%% !!! FINAL SUBMISSIONS.

%%% Copyright (C) 2004-2010 Department of Mathematics, Harvey Mudd College.

%%% See the COPYING document, which should accompany this
%%% distribution, for information about distribution and modification
%%% of the document and its components.

%%% !!! END COPYRIGHT NOTICE.

%%%%%%%%%%%%%%%%%%%%%%%%%%%%%%%%%%%%%%%%%%%%%%%%%%%%%%%%%%%%%%%%%%%%%%
%%% Note that for your Clinic report, you should remove any        %%%
%%% comments that aren't relevant to your report.  You should also %%%
%%% remove the copyright notice assigning copyright to the         %%%
%%% department (depending on your sponsor, you may need to assign  %%%
%%% copyright to the sponsoring organization.                      %%%
%%%%%%%%%%%%%%%%%%%%%%%%%%%%%%%%%%%%%%%%%%%%%%%%%%%%%%%%%%%%%%%%%%%%%%

%%% Preamble.

%%% The top part of your document is called the preamble.  You supply
%%% some basic information about the document (such as its title and
%%% author) in a form that LaTeX can understand here.


%%% The first active line in your LaTeX document is the \documentclass
%%% command, which loads a LaTeX class file.  Class files generally
%%% define the appearance of a document, and include a variety of
%%% structural commands.

%%% Clinic reports use the clinic class, which should be located
%%% somewhere in TeX's search path.

%%% For midyear reports, include the midyear option, as in
\documentclass[midyear]{hmcclinic}
%\documentclass{hmcclinic}

%%% You can also load additional LaTeX packages, or style files, that
%%% affect the way that the document is laid out, typeset, or supply
%%% additional commands or environments.  If you choose to load
%%% additional packages, make sure that they appear *before* the
%%% line loading hyperref; hyperref changes pieces of other
%%% packages, so it's important that it be loaded last.

% \usepackage{graphicx}           % More control over graphic inclusion.
% \usepackage{amsthm}             % AMS theorem styles


%%% Load all other packages before this point.

%%% Load hyperref.
\usepackage[breaklinks=true,
  bookmarks,
  pdfpagemode=UseOutlines,
  pdfpagelayout=SinglePage]{hyperref}


%%% The preamble can also be used to define your own commands and
%%% environments, set some constants that will be used throughout your
%%% document, and so on.

%%% As you may have guessed, LaTeX's comment character is the percent
%%% sign.  Any line that starts with a % will be ignored.  You can
%%% also use the comment character to add comments to the end of a
%%% line that will be parsed by TeX.

%%% The optional \includeonly command allows you to specify the names
%%% of chapters that you want to typeset.  It is useful for debugging
%%% or for working intensely on one particular part of your document
%%% when you don't want to take the time to retypeset the entire document.


%%% This optional command provides additional context around an error.
%%% It can be helpful when tracking down a problem. 
%\setcounter{errorcontextlines}{1000}


%%% Information about this document.

%%% I find it most useful to put identifying information about a
%%% document near the top of the preamble.  Technically, this
%%% information must precede the \maketitle command, which often
%%% appears immediately after the beginning of the document 
%%% environment.  Placing it near the top of the document makes it
%%% easier to identify the document, and keeps it from getting
%%% mixed up with the content of your document.

%%% So, some questions.

%% What is the name of the company or organization sponsoring your project?
\sponsor{Sandia National Laboratories}

%% What is the title of your report?
\title{Parallelizing Intrepid with Kokkos}

%% Who are the authors of the report (your team members)?  (Separate
%% names with \and.)
\author{Alex Gruver \and Ellen Hui \and Tyler Marklyn \and Brett Collins~(Project Manager)}

%% What is your faculty advisor's name?  (Again, separate names with
%% \and, if necessary.)
\advisor{Jeff Amelang}

%% Liaison's name or names?
\liaison{Carter Edwards}

%% By not specifying a date with the \date command, the date the
%% document is typeset will be added.

%% If you need to put in a specific date, do so with
%%  \date{May 13, 2004}
%% You probably shouldn't, however.

%%% End of information section.


%%% New commands and environments.

%%% You can define your own commands and environments here.  If you
%%% have a lot of material here, you might want to consider splitting
%%% the commands and environments into a separate ``style'' file that
%%% you load with \usepackage.

% \newcommand{\coolcommand}[1]{#1 is cool.} % Lets everyone know that
                                % the person or thing that you provide
                                % as the argument to the command is
                                % cool.

% \newcounter{cms}


%%% Some theorem-like command definitions.

%%% The \newtheorem command comes from the amsthm package.  That
%%% package is *not* loaded by the class file, so if you choose
%%% to use these commands, you'll need to load the package above.

% \newtheorem{thm}{Theorem}[chapter]
% \newtheorem{lem}{Lemma}[chapter]


%%% If you find that some words in your document are being hyphenated
%%% incorrectly, you can specify the correct hyphenation using the
%%% \hyphenation command.  Note that words are separated by
%%% whitespace, as shown below.

\hyphenation{ap-pen-dix wer-ther-i-an}

%% Test comment
%%% The start of the document!

%% The document environment is the main environment in any LaTeX
%% document.  It contains other environments, as well as your text.

\begin{document}

%%% The front matter of a large document includes the title page or
%%% pages, tables of contents, lists of figures or tables, and so on,
%%% your abstract, a preface or introduction, and so on.  It's
%%% delineated with the \frontmatter command.



%%% One of the things that the \frontmatter does is make page
%%% numbers appear as lowercase Roman numerals---i, vi, xii, and so
%%% on.

%%% The first thing in the front matter is your title page.  The title
%%% page is formatted by commands in the document class file, so you
%%% don't need to worry about what it looks like -- just putting the
%%% \maketitle command in your document (and filling in the necessary
%%% information for the identification commands above) is enough.

\maketitle

\mainmatter


%%% Content.


\section*{Progress}
% Alex's section
Our main objective for this semester was to rewrite the tensor contraction
functions of Trilinos' Intrepid package to be thread scalable. To do this, we used
Kokkos - a library developed by Sandia to allow simultaneous programming for the
GPU and CPU. 

We began the semester by working through a short online class as well as many
example problems to sharpen our GPU programming skills. Optimizing code for GPU
performance is a career, but by tackling example problems before moving on to
Intrepid we were able to familiarize ourselves with GPU programming enough to
make meaningful contributions to the code.

Since then, we have been able to export multiple kernels from Intrepid and get
them working with Kokkos. We were hoping to achieve speedup on the order of
100x, but since the original intrepid kernels were implemented in a
cache-friendly way, we have changed our expectations to 20-40x speedup for
kernels suited to GPU programming.  On one of the kernels,
contractFieldFieldScalar, we have been able to accomplish a speedup of 35x,
using methods that are reproducible in Kokkos. 

During this time we have learned a lot about using Kokkos, which was one of the
most important aspects of the project, as well as GPU programming in general. We
are now at the point where we understand how to create faster Kokkos versions of
the contraction kernels in Intrepid. However, we have observed some disparity in
the performances of Cuda and Kokkos. Hopefully we'll be able to resolve these
differences going forward.

\section*{Obstacles}
% Ellen's section
Initially, we planned on completing the  fully re-engineered Intrepid package
using Kokkos by early December. While we were not able to finish parallelizing
the entire Intrepid package with sufficient speedup, we plan on delivering a
subset of the kernels by winter break, and giving a seminar on how to best
achieve speedup using Kokkos at our site visit.

Our main obstacle this semester has been the difficulty of reasoning about
memory access patterns on the GPU. Most of our time has been spent determining
general memory access patterns that can be applied to many of the kernels in the
package.

Another obstacle we've encountered is the low ratio of work to memory accesses
inherent in the nature of a few kernels. The kernel that performs dot products,
for example, has to perform one memory access per multiply. Because memory
accesses are so expensive on the GPU, it is incredibly difficult to get
significant speedup on problems like these.

At this point, we have couple prototype kernels producing performance gains on
the expected order of magnitude.  Given the amount of time left in the semester,
it is likely that we will have to settle for presenting a subset of the kernels
instead of a fully reworked Intrepid package in December. 

\section*{Future Plans}
% Brett's section
Finishing the Kokkos versions of Intrepid's ArrayTools functions is our prority
going into the spring.  We were originally hoping to finish this task by Winter
Break (December 16th), but we have come across unexpected problems as mentioned
in the Obstacles section. We plan on focusing our efforts on a couple of the
simpler methods in the ArrayTools library in order to get the desired speedup.
It is more important for this clinic project to demonstrate the capabilites of
Kokkos than it is for it to overhaul the entire codebase of Intrepid.
Furthermore, we believe that once we have correct Kokkos implementations for a
couple of the functions we can apply similar algorithmic techniques to speed up
the remaining functions in Intrepid. 

While working on this project, we have and will continue to record our thoughts
about Kokkos in a location accessible by our liasons.  These notes will give
Sandia valuable feedback about Kokkos and information on how to improve the user
experience. Logging these notes is another of our priorities, since we are
currently serving as Kokkos beta testers. 

Depending on the upcoming work and whether we have a breakthrough in speedup,
we may meet our Winter Break deadline or finish with Intrepid early next
semester. We do not yet know the specifics of our next task, but it will most
likely be creating a Kokkos version of a different high performance computing
package.  Our approach to this next task will be similar to the one we used on
Intrepid. We will begin by finding the simplest functions in the package, then
creating Kokkos versions of those functions and manipulating memory accesses and
parallelization points until we achieve reasonable speedup. Then we will apply
the algorithm to the rest of the functions. Our goal is to finish the Intrepid
package as well as at least one more package by the end of the school year. 

Our stretch goals are creating Kokkos versions of more high performance
computing kernels and integrating the Kokkos versions of the kernels back into
the original packages. We are leaning towards creating Kokkos versions of more
kernels because we will understand Kokkos and what is required to gain speedup.
Integrating the Kokkos versions of the kernels into their original packages
would bring up new challenges that we have not yet faced and therefore would
have a much greater learning curve. 

Overall, our priorities, in order, are: 
\begin{itemize}
	\item Log our experiences with Kokkos on the team wiki
	
	\item Finish creating a Kokkos version of Intrepid

	\item Report on our methodology for writing Kokkos versions of Intrepid
            kernels
	
        \item Repeat this process with a new suite of high performance computing
            kernels
\end{itemize}
We should certainly be able to get the first three of these tasks done by the
end of the clinic project, with the fourth being a reachable stretch goal.

\end{document}
